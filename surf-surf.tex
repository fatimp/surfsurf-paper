\documentclass[reprint,amsmath,amssymb,aps,pre,showkeys,showpacs]{revtex4-1}
% Latex
\usepackage[english]{babel}
\usepackage[utf8]{inputenc}
\usepackage[T1]{fontenc}
% Xelatex
%\usepackage{polyglossia}
%\usepackage{fontspec}
%\setdefaultlanguage{english}
%\setmainfont{DejaVu Sans}
%\setsansfont{DejaVu Sans}
%\setmonofont{DejaVu Sans Mono}
\usepackage{bm}
\usepackage{cleveref}
\usepackage{xcolor}
\usepackage{algpseudocode}
\usepackage{graphicx}
\usepackage{subfigure}
\usepackage[export]{adjustbox}

\definecolor{light-gray}{gray}{0.95}
\newcommand{\code}[1]{\colorbox{light-gray}{\texttt{#1}}}

\begin{document}
\preprint{APS/123-QED}

\author{Aleksei Samarin\textsuperscript{1,2}}
\author{Vasily Postnicov\textsuperscript{1}}
\author{Marina Karsanina\textsuperscript{1}}
\author{Efim Lavrukhin\textsuperscript{1,2}}
\author{Dina Gafurova\textsuperscript{3}}
\author{Aleksey Khlyupin\textsuperscript{1}}
\author{Kirill Gerke\textsuperscript{1}}
\email{kg@ifz.ru}

\affiliation{\textsuperscript{1}Schmidt Institute of Physics of the Earth of
  Russian Academy of Sciences, Moscow, 107031, Russia}
\affiliation{\textsuperscript{2}Computational Mathematics and Cybernetics,
  Lomonosov Moscow State University, Moscow, 119991, Russia}
\affiliation{\textsuperscript{3}Oil and Gas Research Institute Russian Academy
  of Sciences (OGRI RAS) 3, Gubkina st., Moscow, 119333, Russian Federation}

\title{Robust surface correlation functions evaluation from experimental
  discrete digital images}

\begin{abstract}
  Correlation functions (CFs) are universal structural descriptors;
  surface-surface $F_{ss}$ and surface-void $F_{sv}$ CFs are a subset containing
  additional information about the interface between the phases. The description
  of the interface between pores and solids in porous media is of particular
  importance and recently Ma and Torquato \cite{ma2018SS} proposed an elegant
  way to compute these functions for a wide variety of cases. However, their
  "continuous" approach is not always applicable to digital experimental 2D and
  3D images of porous media as obtained using X-ray tomography or scanning
  electron microscopy due to non-singularities in chemical composition or local
  solid material's density, and partial volume effects. In this paper we propose
  to use edge-detecting filters to compute surface CFs in the "digital" fashion
  directly in the images. Computed this way, surface correlation functions are
  the same as analytically known for Poisson disks in case the resolution of the
  image is adequate. Based on the multi-scale image analysis we developed a
  $C_{0.5}$ criterion that can predict if the imaging resolution is enough to
  make an accurate evaluation of the surface CFs. We also showed that in case
  when the input image contains all major features, but do not pass the
  $C_{0.5}$ criterion, it is possible with the help of image magnification to
  sample CFs almost similar to those obtained for high-resolution image of the
  same structure with high $C_{0.5}$. The computational framework as developed
  here is open source and available within the \code{CorrelationFunctions.jl}
  package developed by our group. Our "digital" approach was applied to a wide
  variety of real porous media images of different quality. We discuss critical
  aspects of surface correlation functions computations as related to different
  applications. The developed methodology allows applying surface CFs to
  describe the structure of porous materials based on their experimental images
  and enhance stochastic reconstructions or super-resolution procedures, or
  serve as an efficient metrics in machine learning applications due to
  computationally effective GPU implementation.
\end{abstract}
\keywords {3D images; correlation functions; surface-surface function;
  surface-void function; image analysis; image scaling}

\maketitle

\section{Introduction}
\label{sec:intro}
Correlation functions are invaluable universal descriptors of structure
and in this context are utilized in a multitude of scientific disciplines:
material sciences \cite{Cecen,chen2022quantifying}, rock
\cite{ledesma2018effect} and soil physics
\cite{Euras2012,PLoS_ONE,KarsaninaEJSS}, cosmology \cite{TakadaJain} and food
engineering \cite{Derossi2019}, to name just a handful. As such, CFs were used
to characterize the morphology \cite{tensorPRE} and representativeness via
correlation lengths \cite{vcapek2011transport,thovert2011grain}, compare
structures \cite{jiao2014chawla,EPL1,REVpaper}, compress structural information
\cite{SciRep1,Havelka,KarsaninaEJSS}, describe structural dynamics
\cite{chen2015dynamic,PLoS_ONE,xu2022correlation}, extract features for deep
learning \cite{Miao2017,kamrava2020linking,roding2020predicting,KarsaninaEJSS},
perform stochastic reconstructions
\cite{Adler_recon,Y-T,EPL2,tahmasebiPRL,karsaninaPRL} and fuse multi-scale
images \cite{SciRep1,chen2016stochastic,Geoderma2018}. Stochastic reconstruction
is a special topic of interest, as this approach allows to solve an inverse
problem and recover structure from a known set of correlation functions –- and
this ability for recovery is the basis for majority of potential usages in the
list above. Early reconstruction techniques mainly involved two-point
probability $S_2$ function, but were improved to include lineal $L_2$ function
\cite{Y-T,vcapek2009stochastic,vcapek2011transport}, cluster $C_2$ function
\cite{JiaoPNAS,jiao2014chawla} and potentially surface $F_{SS}$ and $F_{SV}$
functions. With the proper handling \cite{EPL2} it is possible to incorporate
numerous CFs into a reconstruction procedure. Increasing the number (and order)
of functions leads to higher information content of the CFs set
\cite{Gommes1,Gommes2} and, thus, allows reconstructing and performing
stationarity/representativeness characterization \cite{REVpaper} for structures
of any complexity.

There are two ways to obtain correlation functions for a given structure at hand
-- either measure them experimentally with the help of scattering intensity
\cite{debye1957scattering,li2018accurate}, or measured from images. The first
approach suffers from the limitation of CFs that can be obtained this way
\cite{Gommes_notchord}, while the second one provides information with limited
resolution or/and resolution to field-of-view ratio \cite{SciRep1}. Moreover,
the most useful imaging methods such as X-ray computed tomography (XCT) and
scanning electron microscopy (SEM) due to their underlying physical principles
provide gray-scale images which, as we shall argue later on, need to be
segmented into constituent phases before CFs computation. It is important to
note that such gray-scale images do not represent spatial distribution of the
phases, but, for example, X-ray spatial attenuation or electron back-scattering
intensities. However, if properly segmented, high-resolution digital images do
provide a possibility to compute any correlation function. Experimental
measurements using small angle scattering (SAS) are traditionally used to
evaluate $S_2$, but it is also possible to relate scattering intensities to
surface correlation functions \cite{dietrich1995scattering,ma2018SS}. It would
be of great practical importance to obtain surface correlation functions from
both SAS and imaging (focused ion beam milling combined with SEM allows to
obtain the finest imaging resolution and sample CFs from
polished surfaces as opposed to surface imaging \cite{FIB-SEMpaper}) to estimate
coefficients for surface and bulk scattering (see Eq.13 in \cite{ma2018SS}). But
there is a catch -- SAS possesses close to infinite surface resolution as
opposed to digital pixelized images with inherent segmentation problems due to
partial volume effects, something we discuss extensively in this paper.

Recently, Ma and Torquato \cite{ma2018SS} laid foundation to precise surface
correlation function computations and showed the usefulness of these CFs
evaluation for numerous problems. They have implemented an elegant solution with
the help of infinite resolution random fields that (if the fields are
thresholded) allows finding an exact intersection with the sampling line. In
addition, they also showed that surface CFs for digital images can be computed
by converting integer fields (i.e., location of the phases) to float fields with
the help of Gaussian filter \cite{ma2018SS}. Unfortunately, their solution is
not readily applicable to the majority of grayscale images due to
non-singularity in their chemical constitution. The reason is the difference
between the contour and real interface between different phases within the
material due to the partial volume effects \cite{Wildenschild_Sheppard} for XCT
imaging -- the presence of multiple phases within the same pixel/voxel. In other
words, the attenuation is a function of both density and atomic number, which
usually are distributed non-uniformly below the XCT imaging resolution. Somewhat
similar is also relevant for SEM imaging, as secondary or back-scattered
electrons are effectively a convolution of partial signals coming from different
depths \cite{Bultreys_review}. The exact sub-voxel thresholding based on
gray-scale image (similar to the technique used in \cite{ma2018SS}) is only
available in case it is monomineral, i.e., the solid phase consists of a
completely chemically homogeneous substance that contrasts perfectly with
air/vacuum filled pore phase -- this is rarely the case for natural
materials. Moreover, even if the image of a monomineral sample (in natural
porous media carbonate rocks can be considered monomineral depending on their
genesis) has sub-resolution features on XCT image, while it is possible to
compute the porosity of the voxel, it's still not possible to extract the
interface. All these and additional problems (such as, for example, experimental
noise and artifacts from inverse Radon transform) arising during gray-scale
image processing as related to image segmentation were extensively discussed
elsewhere \cite{NNseg}.
%In other words, simple thresholding of the grey-scale image to compute surface
%functions using the approach proposed in
%\cite{ma2018SS} (their section V.C) may result in inaccurate surface
%identification and surface functions are not defined for grey-scale %images
%(due to possible sub-resolution porosity).
Thus, we argue, that the usage of current
state-of-the-art segmentation techniques is necessary prior to the conversion
to a coarse-grained scalar field. Moreover, finding an intersection between a
line segment and the contour is a computationally expensive procedure that
prevents on-the-fly application of surface functions. If a binary image
(consisting of pores and solids) is produced by adequate segmentation technique,
we still lack a robust and computationally effective procedure to evaluate
surface CFs from experimental XCT and SEM images of natural heterogeneous
materials, for instance, rocks and soils.

\begin{figure}[ht]
  \centering
  \includegraphics[width=\linewidth]{images/scheme.png}
  \caption[]{A schematic depiction of a binary porous media (pores are shown in
    color) with examples of positive events for surface $F_{SS}$, $F_{SV}$ and
    two-point probability $S_2$ correlation functions. The zoomed in area
    represents the difference between the true ``continuous'' interface in
    between pore and solid phases with pixelized ``digital'' interface emerging
    due to limited resolution of digital images.}
  \label{fig:scheme}
\end{figure}

To compute surface functions one first needs to elucidate the interface between
two phases (we shall consider only pores and solids, but, obviously, the
computations can be performed to multi-phase systems in the similar manner). The
interface area has infinitesimal volume, but its location on the digital image
is not easy due to the pixelization of the interface between the binary phases
\cite{ma2018SS}. By adopting the interface width $\varepsilon$ = pixel/voxel
size one can create a very crude approximation of the real interface, but the
usage of interfaces between voxels ``as is'' leads to known problems in surface
area and surface geometry evaluation, in application of 3D imaging for geometry
and topology analysis \cite{AWR_PNM,PNM_Morse} or energy minimization problems
\cite{frank2018energy}. Another option would be to describe the boundaries
between voxels with some curves, e.g., splines. This way it would be possible to
perform surface CFs sampling by line intersection as described by Ma and
Torquato \cite{ma2018SS}. While an exact boundary can be obtained for
deterministic structures such as disk packings, splines would provide only a
very approximate solution for the boundaries of arbitrary digitized structure
due to the limit in resolution for a given image \cite{gerke2012tomographic}. In
other words, contours extracted from digital images will approach real
boundaries between phases only when the spatial resolution approaches
infinity. But the same is also true for digital pixelized/voxelized
images. Thus, the options to compute surface correlation functions include
``continuous'' approach (such as implemented in \cite{ma2018SS}) and ``digital''
approach (similar to computation of $S_2$ from digital images) as depicted in
\cref{fig:scheme}. Here we propose a modified ``digital'' approach that lies in
between the exact and pixelized solutions for a number of reasons:
\begin{enumerate}
  \item For a binarized XCT or SEM image the interpolation of the interface
    using "continuous" or "digital" in the limit of $\varepsilon \to 0$ would
    produce the same results. Thus, as it is not possible to obtain the exact
    continuous interface from such an image, using "digital" approach is natural
    as applied to digital pixel/voxel images.
  \item "Continuous" approach is expensive numerically, as it requires to find the
    intersections between a line and a curve.
  \item Digital approach allows utilizing linear scan CFs computation or the
    Fast Fourier Transform (FFT) -- the latter is the fastest way to evaluate
    full CF maps (not for all functions thought) and is very efficient on modern
    GPUs.
\end{enumerate}

In this paper we build upon foundational work of Ma and Torquato \cite{ma2018SS}
and develop a robust and efficient approach to compute surface correlation
functions from digital 2D and 3D images. The rest of the manuscript is organized
as follows: in \cref{sec:details} we provide all methodological details for
surface CFs computation including analytical solutions to verify the proposed
methodology and describe an image library for extensive testing of our
algorithms, \cref{sec:results} presents all major results of surface functions
evaluations. We discuss obtained results, including the effects of image
scaling, and outline future uses of surface functions within
\cref{sec:outline}. The paper concludes with a summary in \cref{sec:summary}.

\section{Theoretical background}
\label{sec:theory}
\subsection{Correlation functions and definitions}
\label{sec:definitions}
Firstly, we introduce an indicator function $I^{(i)}(\mathbf{x})$ which marks
pixels of 2D and voxels of 3D digitized images as belonging to the phase $i$ or
not. It can be defined as:
\begin{equation*}
  I^{(i)}(\mathbf{x}) = \left\{
  \begin{array}{ll}
    1 & \quad \mathbf{x} \in V_i \\
    0 & \quad \text{otherwise}
  \end{array}
  \right.
\end{equation*}
where $V_i \subset \mathbb{R}^n$ is the region occupied by phase $i$. For
statistically homogeneous media the ensemble average of $I^{(i)}$ equals volume
fraction of a given phase. For binary media the following equality holds:
\begin{align*}
  \phi_{void} &+ \phi_{solid} = 1 \\
  \phi_i &= \langle I^{(i)}(\mathbf{x}) \rangle
\end{align*}
In a similar fashion we can define an interface indicator function
$M(\mathbf{x})$ which provides interface area $s$ if averaged over the whole
image:
\begin{align}
  M(\mathbf{x}) = |\nabla I^{(solid)}(\mathbf{x})| &= |\nabla I^{(void)}(\mathbf{x})|
  \label{eq:interface} \\
  \langle M(\mathbf{x}) \rangle &= s
  \label{eq:surface}
\end{align}
The simplest, yet foundational correlation function is the two-point probability
function $S_2$ which is defined as a probability that the ends of a random line
segment belong to the same phase:
\begin{equation}
  S_2^{(i)}(\mathbf{x}_1, \mathbf{x}_2) = \langle I^{(i)}(\mathbf{x}_1)
  I^{(i)}(\mathbf{x}_2) \rangle \label{eq:twopoint}
\end{equation}
This equation can be further simplified for statistically homogeneous media, as
$S_2$ will dependent only on the relative displacement $\mathbf{r}$:
\begin{equation*}
  S_2^{(i)}(\mathbf{x}_1, \mathbf{x}_2) = S_2^{(i)}(\mathbf{r})
\end{equation*}
The value of $S^{(i)}_2(\mathbf{r})$ at $\mathbf{r} = 0$ is a fraction of phase
$i$ in a medium:
\begin{equation*}
  S_2^{(i)}(0) = \phi_i
\end{equation*}
For isotropic media a vector displacement $\mathbf{r}$ can be replaced with its
length $r = |\mathbf{r}|$ and $S_2(\mathbf{r})$ with $S_2(r) = S_2(|\mathbf{r}|)$.
Now, analogously to $S_2$ we can define surface-surface and surface-void
correlation functions (for homogeneous and isotropic media straight away):
\begin{align}
  F_{ss}(r) &= \langle M(x)M(x+r) \rangle \label{eq:fss} \\
  F_{sv}(r) &= \langle M(x)I^{(void)}(x+r) \label{eq:fsv} \rangle
\end{align}

The interface indicator function $M(x)$ is basically defined as infinity on the
interface and zero elsewhere. Therefore, it is hard to directly compute
$\langle M(x), M(x + r) \rangle$. We can replace the line of zero width and
infinite value with a strip of width $\varepsilon$ and value
$\frac{1}{\varepsilon}$: $M(x; \varepsilon)$. Then
$F_{SS}(x; \varepsilon) = \langle M(x; \varepsilon), M(x + r; \varepsilon) \rangle$
is simply an intersection area of two strips -- original and shifted --
multiplied by $(\frac{1}{\varepsilon})^2$.
And with $\varepsilon \to 0 \Rightarrow F_{SS}(r; \varepsilon) \to F_{SS}(r)$.
Area between two black dashed circles on \cref{fig:Fss-explained} represents
$M(x; \varepsilon)$, red is for $M(x + r; \varepsilon)$; blue color marks the
intersection area. The treatment of the interface in the case of $F_{SV}$ is
similar, but this function is not auto-correlation, but rather cross-correlation
of the interface with the void phase. More detailed information on two-point
probability and surface correlation functions can be found in the comprehensive
Torquato's book \cite{Torquato_book}.

\begin{figure}
  \centering
  \includegraphics[width=\linewidth]{images/Fss.png}
  \caption[]{Interpretation of $F_{SS}$ as a self-intersection of the interface.}
  \label{fig:Fss-explained}
\end{figure}

\subsection{Analytical solutions}
For Poisson disks and balls one can derive exact analytical surface-surface and
surface-void functions. For overlapping disks with radius $R$ and centers
generated by Poisson point process with parameter $\lambda$ we have (see the
derivation of these formulas in \cref{ap:overlapping-disks}):
\begin{align}
  F_{SV}(r) &= S_2(r) \left\{
  \begin{array}{ll}
    2(\pi - B)R \lambda & \quad r<2R \\
    2\pi R \lambda & \quad \text{otherwise}
  \end{array} \right. \label{eq:fsv_final} \\
  F_{SS}(r) &= S_2(r) \left\{
  \begin{array}{ll}
    \frac{(2(B-\pi)R\lambda)^2Ar + 4\sqrt{A}R^2\lambda}{Ar} & \quad r<2R \\
    (2\pi R\lambda)^2 & \quad \text{otherwise},
  \end{array} \right. \label{eq:fss_final}
\end{align}
where $S_2$ is the regular two-point correlation function and
\begin{align*}
  A &= 4R^2 - r^2 \\
  B &= \arccos(\frac{r}{2R}).
\end{align*}

For 3D balls the relationships are readily available in the literature
\cite{Torquato_book,ma2018SS}:
\begin{align*}
  F_{sv}(r) &= S_2(r) \left\{
  \begin{array}{ll}
    4\pi R^2\lambda(\frac{1}{2} + \frac{r}{4R}) & \quad r<2R \\
    4\pi R^2\lambda & \quad \text{otherwise}
  \end{array} \right. \\
  F_{ss}(r) &= S_2(r) \left\{
  \begin{array}{ll}
    {(4\pi R^2 \lambda (\frac{1}{2} + \frac{r}{4R}))^2 + \frac{2\pi R^2 \lambda}{r}} & \quad r<2R \\
    (4\pi R^2 \lambda)^2 & \quad \text{otherwise}.
  \end{array} \right.
\end{align*}

These analytical solutions will be used to verify the accuracy of our evaluation
of surface CFs for such systems.

\section{Methodological details}
\label{sec:details}
By looking at \cref{eq:twopoint,eq:fss,eq:fsv} one can observe
some significant similarities between two-point probability and two-point
surface functions (see \cref{fig:scheme}). The $S_2$ can be viewed as a
auto-correlation of the image, i.e., computing correlations between shifted
realization of the image -- something that can be used to effectively compute
$S_2$ with the help of the FFT on modern hardware, especially GPUs. If we apply
the same analogy to $F_{SS}$, \cref{eq:fss} can be considered as an intersection
of the interface with itself for all possible shifts. Recall that for the
correlation length of zero this intersection is technically infinity
(\cref{sec:definitions}). While surface-void function $F_{sv}$ is well-defined
at $r=0$, it has some notable features. It can be calculated as
cross-correlation between the interface and the void phase. Because
cross-correlation is not commutative there is a complementary function,
void-surface function $F_{vs}$, which is defined as cross-correlation between
the void phase and the interface. Because these two functions are dependent,
namely $F_{sv}(\mathbf{x}) = F_{vs}(-\mathbf{x})$, in this paper we will only
speak of $F_{sv}$ function.

\subsection{The general algorithm}
\label{sec:general}
While we present and compare a bunch of slightly different methods to evaluate
surface CFs, they are all based on a single general algorithm for each surface
function as described below ($A$ refers to the input 2D or 3D digital image, $M$
is the interface between two phases, and $V$ is the image of the void phase
only):

$F_{SS}$ algorithm:
\begin{algorithmic}[1]
  \Procedure{surfsurf}{$A, i$}
  \State $A' \gets I^{(i)}(A)$
  \Comment{Apply indicator function to $A$.}
  \State $M \gets M(A')$
  \Comment{Extract interface from $A'$.}
  \State \textbf{return} $\star(M, M)$
  \Comment{Auto-correlation of $M$.}
  \EndProcedure
\end{algorithmic}

$F_{SV}$ algorithm:
\begin{algorithmic}[1]
  \Procedure{surfvoid}{$A, i$}
  \State $A' \gets I^{(i)}(A)$
  \Comment{Apply indicator function to $A$.}
  \State $M \gets M(A')$
  \Comment{Extract interface from $A'$.}
  \State $V \gets I^{(void)}(A')$
  \Comment{Extract void phase from $A'$.}
  \State \textbf{return} $\star(M, V)$
  \Comment{Cross-correlation of $M$ and $V$.}
  \EndProcedure
\end{algorithmic}

The algorithm is very general and allows utilizing different techniques for
interface extraction or computation of the correlations, as detailed below. The
cross-correlation function is introduced in \cref{sec:cross-comp}. Consecutive
stages of our method are shown in \cref{fig:stages}. The programming
implementation details are provided in the \cref{ap:b_code}.

\subsection{Interface extraction}
We shall consider two methods for the interface extraction -- the first is a
naïve approach, while the second is the one we propose to routinely apply for
XCT and SEM images analysis.

\begin{figure*}[!pt]
  \centering
  \includegraphics[width=0.6\linewidth]{images/algo.png}
  \caption[]{Schematic example of the algorithm for surface
  functions computation (2D case is considered for visibility).}
  \label{fig:stages}
\end{figure*}

\textit{Distance map}.
One solution to extract an interface between phases from an image is to use the
distance map (also called the distance transform). Distance transform is a
mapping from a binary image to a scalar field using one of the following
functions:
\begin{align}
  D_{o}(\mathbf{x}) &= \left\{
  \begin{array}{ll}
    0 & \quad \mathbf{x} \in V_{void} \\
    \min \rho(\mathbf{x}, \mathbf{y}) \ \forall \mathbf{y} \in V_{void} & \quad \text{otherwise}
  \end{array}
  \right. \label{eq:dist-outer} \\
  D_{i}(\mathbf{x}) &= \left\{
  \begin{array}{ll}
    0 & \quad \mathbf{x} \not\in V_{void} \\
    \min \rho(\mathbf{x}, \mathbf{y}) \ \forall \mathbf{y} \not\in V_{void} & \quad \text{otherwise}
  \end{array}
  \right. \label{eq:dist-inner}
\end{align}
where $V_{void}$ is a set containing elements belonging to the void phase and
$\rho(\mathbf{x}, \mathbf{y})$ is the Euclidean distance between points
$\mathbf{x}$ and $\mathbf{y}$. Now $M(\mathbf{x})$ in \cref{eq:interface} may be
written as follows:
\begin{equation*}
  M(\mathbf{x}) = \left\{
  \begin{array}{ll}
    1 & 0 < \quad D(\mathbf{x}) \le \sqrt{d} \\
    0 & \quad \text{otherwise}
  \end{array}
  \right.
\end{equation*}
where $d$ is dimensionality of the image ($d = 2$ for 2D image, $d = 3$ for 3D
image) and $D(\mathbf{x})$ is evaluated using either \cref{eq:dist-outer} or
\cref{eq:dist-inner}. When $D = D_i$ we call the resulting interface
``the inner interface'' and when $D = D_o$ we call it ``the outer
interface''. Hence the inner interface consists of elements in the void phase
which lie close to the solid phase and the outer interface is the other way
around, i.e. consists of elements in the solid phase which lie close to the void
phase.

\textit{Image filtering}. This method creates a grayscale image of the interface
with the help of the edge-detecting filter. The filter works by convolving the
input image with signals \cref{eq:filter-2d} for 2D case, or
\cref{eq:filter-3d-1}-\cref{eq:filter-3d-3} for 3D case.
\begin{align}
  F &= \frac{\sqrt{2}}{9} \left[
    \begin{array}{ccc}
        1 & 1 & 1 \\
        1 & -8 & 1 \\
        1 & 1 & 1
      \end{array}
    \right] \label{eq:filter-2d} \\
  F_{ij0} &= \frac{\sqrt{2}}{27} \left[
    \begin{array}{ccc}
        1 & 1 & 1 \\
        1 & 1 & 1 \\
        1 & 1 & 1
    \end{array}
    \right] \label{eq:filter-3d-1} \\
  F_{ij1} &= \frac{\sqrt{2}}{27} \left[
    \begin{array}{ccc}
      1 & 1 & 1 \\
      1 & -26 & 1 \\
      1 & 1 & 1
    \end{array}
    \right] \label{eq:filter-3d-2} \\
  F_{ij2} &= F_{ij0} \label{eq:filter-3d-3}
\end{align}
These signals are so called high-pass finite impulse response
filters. Coefficients in tensors \cref{eq:filter-2d} and
\cref{eq:filter-3d-1}-\cref{eq:filter-3d-3} sum to zero, hence a result of
application of an edge-detecting filter to a constant input is also zero, which
means there is no ``interface'' in a constant image. On the other side, when an
input image varies a lot, application of this filter gives a big response in the
place of variation, effectively detecting an edge. Scaling coefficients in front
of the tensors are found empirically to achieve a good correspondence with a
theory.

An application of these filters (i.e. convolution) can be easily parallelized on
GPU, and we do this by writing native Cuda code. After the convolution step is
finished we take absolute values of the result as an approximation for $M(A')$
(recalling that $A' = I^{(void)}(A)$ where $A$ is the input image). In physical
sense the result of this procedure is proportional to the probability that a
pixel/voxel under consideration belongs to the interface. From now on we shall
call the filter $F$ simply ``edge-detecting filter'' when no other filter is
explicitly mentioned.

\subsection{Computing correlations}
\label{sec:cross-comp}
Let us now describe the approach for the computation of cross-correlation
function between two images $f$ and $g$. In the case of auto-correlation
function we assume $f = g$. Algorithm is straightforward when boundary
conditions are periodic. For non-periodic boundary conditions it is slightly
more involved.

\textit{Periodic boundary condition}
\begin{algorithmic}[1]
  \Procedure{$\star$}{$f, g$}
  \State $\hat{f} = F(f)$
  \State $\hat{g} = F(g)$
  \Comment Compute FFT of the input.
  \State $\hat{cc} \gets \hat{f} \cdot \overline{\hat{g}}$
  \Comment Multiply element-wise.
  \State $cc \gets F^{-1}(\hat{cc})$
  \Comment Compute IFFT of $\hat{cc}$.
  \State \textbf{return} $cc$ divided by the number of pixels/voxels in the input.
  \EndProcedure
\end{algorithmic}

\textit{Non-periodic boundary condition}
\begin{algorithmic}[1]
  \Procedure{$\star$}{$f, g$}
  \State Pad $f$ and $g$ with zeros in each dimension to the size $2n-1$ where
  $n$ is the size of the image in that dimension.
  \State $\hat{f} = F(f)$
  \State $\hat{g} = F(g)$
  \Comment Compute FFT of the input.
  \State $\hat{cc} \gets \hat{f} \cdot \overline{\hat{g}}$
  \Comment Multiply element-wise.
  \State $cc \gets F^{-1}(\hat{cc})$
  \Comment Compute IFFT of $\hat{cc}$.
  \State circle shift $cc$ by $n - 1$ in each direction, so index range becomes
  from $-(n - 1)$ to $n - 1$.
  \State $cc_{ij} \gets cc_{ij} / q_{ij}$ where $q_{ij} = (n - |i|)(m - |j|)$
  \Comment Divide each element $cc_{ij}$ by the number of pixels/voxels whose
  coordinate difference is equal to $(i, j)$.
  \State \textbf{return} $cc$.
  \EndProcedure
\end{algorithmic}

The results of the computations are a full correlation map. To obtain the
ensemble averaged surface correlation functions one just needs to convert the
map to CF vs. $r$ relationship. This is also possible to compute directional
surface CFs in this manner in case the structure at hand is anisotropic. Without
loss of generality, $\star$($f, g$) can be also computed by scanning with the
line segment of length $r$, but this approach is more efficient on CPU as
compared to GPU implementation here. If not stated otherwise, we report the
ensemble averaged functions computed from the full map (assuming the isotropy of
the input image). One can easily compute directional CFs
\cite{jiao2014chawla,EPL1} using both segment scanning or from the correlation
maps.

\section{Application to synthetic images, verification and betterment of the methodology}
\label{sec:results}
\subsection{Comparisons against analytical solutions}
\label{sec:comparison}
To evaluate the accuracy of surface CFs computations the most straightforward
way is to compare them against analytical solutions. We use classical
analytical solutions for overlapping disks and balls with fixed radius $R$ and
centers generated by Poisson point process with parameter $\lambda$. Starting
with an image with resolution of $4096 \times 4096$ pixels, we then downscale it
by 4, 16, and 64 times with the help of bicubic interpolation
\cite{ledesma2018effect} (see \cref{fig:disks-noise}). We show that calculated
correlation functions approach theoretical values when the resolution is high
enough.
\begin{figure*}[!pt]
  \centering
  \subfigure[Image size $4096 \times 4096$, disk radius 61.44 pixels, $C_{0.5} = 0.9593$]{
    \includegraphics[width=0.22\linewidth, frame]{images/disks-0015-5e-5-4096.png}
    \label{fig:disks-4096}}
  \hfill
  \subfigure[Image size $1024 \times 1024$, disk radius 15.36 pixels, $C_{0.5} = 0.9185$]{
    \includegraphics[width=0.22\linewidth, frame]{images/disks-0015-5e-5-1024.png}
    \label{fig:disks-1024}}
  \hfill
  \subfigure[Image size $256 \times 256$, disk radius 3.84 pixels, $C_{0.5} = 0.8356$]{
    \includegraphics[width=0.22\linewidth, frame]{images/disks-0015-5e-5-256.png}
    \label{fig:disks-256}}
  \hfill
  \subfigure[Image size $64 \times 64$, disk radius 0.96 pixels, $C_{0.5} = 0.6549$]{
    \includegraphics[width=0.22\linewidth, frame]{images/disks-0015-5e-5-64.png}
    \label{fig:disks-64}}
  \vskip\baselineskip
  \subfigure[Image size $4096 \times 4096$, value noise, $C_{0.5} = 0.9692$]{
    \includegraphics[width=0.22\linewidth, frame]{images/noise-4096.png}
    \label{fig:noise-4096}}
  \hfill
  \subfigure[Image size $1024 \times 1024$, value noise, $C_{0.5} = 0.9386$]{
    \includegraphics[width=0.22\linewidth, frame]{images/noise-1024.png}
    \label{fig:noise-1024}}
  \hfill
  \subfigure[Image size $256 \times 256$, value noise, $C_{0.5} = 0.8770$]{
    \includegraphics[width=0.22\linewidth, frame]{images/noise-256.png}
    \label{fig:noise-256}}
  \hfill
  \subfigure[Image size $64 \times 64$, value noise, $C_{0.5} = 0.7524$]{
    \includegraphics[width=0.22\linewidth, frame]{images/noise-64.png}
    \label{fig:noise-64}}
  \caption[]{2D images of overlapping disks
    (\subref{fig:disks-4096}-\subref{fig:disks-64}) and value noise
    (\subref{fig:noise-4096}-\subref{fig:noise-64}) with different resolution
    (\cref{sec:results}).}
  \label{fig:disks-noise}
\end{figure*}

It is easy to show that for $a > 0$:
\begin{align}
  a^2 F_{SS}(a \mathbf{x}, a(\mathbf{x} + \mathbf{r})) &= F_{SS}(\mathbf{x},
  \mathbf{x} + \mathbf{r}) \label{eq:scale-ss} \\
  a F_{SV}(a \mathbf{x}, a(\mathbf{x} + \mathbf{r})) &= F_{SV}(\mathbf{x},
  \mathbf{x} + \mathbf{r}) \label{eq:scale-sv}
\end{align}
Assuming that our images represent homogeneous and isotropic media, we calculate
$F_{SS}(ar)$ and $F_{SV}(ar)$ for each original and rescaled image and multiply
it by coefficient $a^2 = (L_{scaled}/L_{orig})^2$ and $a = L_{scaled}/L_{orig}$
respectively, where $L_{scaled}$ is the side of the rescaled image and
$L_{orig}$ is the side of the original image. The resulting scaled surface CFs
are shown in \cref{fig:scaling-alltogether}.

\begin{figure*}[!pt]
  \centering
  \subfigure[Surface-surface, disks]{
    \includegraphics[width=0.475\linewidth]{images/plot-ss-balls.png}
    \label{fig:fss-scaling}}
  \hfill
  \subfigure[Surface-void, disks]{
    \includegraphics[width=0.475\linewidth]{images/plot-sv-balls.png}
    \label{fig:fsv-scaling}}
  \vskip\baselineskip
  \subfigure[Surface-surface, value noise]{
    \includegraphics[width=0.475\linewidth]{images/plot-ss-noise.png}
    \label{fig:fss-scaling-noise}}
  \hfill
  \subfigure[Surface-void, value-noise]{
    \includegraphics[width=0.475\linewidth]{images/plot-sv-noise.png}
    \label{fig:fsv-scaling-noise}}
  \caption[]{Surface-surface and surface-void correlation functions for
    images of overlapping disks and thresholded value noise obtained with
    different resolution. Theoretical results are also shown for overlapping
    disks. Image filtering was used for interface extraction.}
  \label{fig:scaling-alltogether}
\end{figure*}

We observe two effects when downscaling the original image. The
first effect is that $F_{SS}(ar)$ and $F_{SV}(ar)$ for downscaled images lack in
detail (i.e., they are less ``noisy''). This can be easily explained by the
interface between phases becoming ``simpler'' when resolution decreases. The
second and much more important effect is that both correlation functions become
largely underestimated with the decline in resolution. This latter effect is in
general connected to the fact that the interface in digital images has
non-negligible thickness -- i.e., the difference between real ``continuous''
interface versus ``digital'' interface (\cref{fig:scheme}). It is logical to
assume that with increasing spatial resolution the influence of this thickness
will diminish. This is exactly what we observe on \cref{fig:scaling-alltogether}
where increasing disk discretization leads to convergence with analytical
solution; the accuracy of the computed CFs is almost perfect for discretization
of about 62 pixels for each disk's diameter. A natural question arises: ``Is
there a criterion of image quality (resolution) that allows to predict the
quality of surface CFs evaluation from digital images?''. Turns out there is a
possibility to establish such an empirical criterion based on spectral analysis
of input images and \cref{sec:crit} will provide all necessary details.

\subsection{Convergence of correlation functions with increasing resolution}
\label{sec:value-noise}
Now consider a situation when an image is obtained by taking samples of a function
$f: \mathbb{R}^n \rightarrow \left\{0, 1\right\}$. The samples are taken from a
regularly spaced lattice which covers the range $[0, L]^n$  with interval
$\Delta$ between samples. The resulting image has then $L/\Delta + 1$ pixels in
each dimension. We will show empirically that for any sequence $\Delta_k$ so
that $\Delta_{k+1} < \Delta_k$ and $\Delta_k \rightarrow 0$ sequences
$F_{SS}^{\Delta_k}(r)$ and $F_{SV}^{\Delta_k}(r)$ computed for images with
lattice interval $\Delta_k$ converge to some limit functions $F_{SS}(r)$ and
$F_{SV}(r)$.

An example of $f$ is a thresholded value noise function. A value noise is a
procedurally generated noise which in one-dimensional case works as
follows. Firstly define a family of regularly spaced lattices $U_k$ with points
in coordinates $2^{-k}n,\ n \in \mathbb{Z}$. Each point of a lattice is assigned
a random value in the range $[0, 1]$. Then for each $U_k$ we define a
function $g_k(x): \mathbb{R} \rightarrow \mathbb{R}$ which interpolates random
values in $U_k$ linearly. This value noise function is defined as follows:
\begin{align*}
  g(x; n) &= \frac{\sum\limits_{k=0}^{n-1} 2^{-k}g_k(x)}{\sum\limits_{k=0}^{n-1}
    2^{-k}} \\
  &= \frac{2^n}{2^{n+1}-2} \sum\limits_{k=0}^{n-1} 2^{-k}g_k(x)
\end{align*}
Here $n$ is a parameter called the number of octaves. A thresholded noise is then
\begin{align*}
  f(x; n) = \Theta(g(x; n) - 1/2)
\end{align*}
where $\Theta(x)$ is the Heaviside step function. A generalization for higher
dimensionality is constructed using the sample principle. An example of
thresholded value noise with three octaves is depicted on
\cref{fig:disks-noise}. Correlation functions calculated for the noise with
$n = 3$ at different resolutions are on \cref{fig:scaling-alltogether}. If we
calculate surface-surface and surface-void correlation functions for these
images and scale them as described above, we again converge to some ``true''
correlation functions for continuous function $f$.

\subsection{Effects of image magnification}
\label{sec:magnify}
As we have just observed for analytical Poisson disks and arbitrary structures
based on thresholded value noise, with increasing resolution (or in other words
-- decreasing coarseness) of the image, surface correlation functions converge
to "infinite resolution" values that can be seen as a target of CFs
computations. On the other hand, if the image at hand is of limited
resolution, it would be very practical to know if rescaling this image with
increased resolution can lead to acceptable surface-surface functions
evaluation. To explore such a possibility we tried different methods to magnify
the image: bilinear, bicubic and nearest-neighbor interpolation.

To have true CFs for a comparison, we again utilize analytical Poisson disks: a
low-resolution and high-resolution images were produced by discretization of
disks at known positions, and then the low-resolution image was
magnified. Because the resized image becomes grayscale, we use the following
function as an indicator for the void phase provided the void phase is marked as
0 in the input image:
\begin{equation*}
  I^{(void)}(x) = (1 - x)^{1.5}.
\end{equation*}

We found that all interpolation methods are good for resizing purposes with
exception of the nearest-neighbor approach. As an example, \cref{fig:resized}
shows the images of overlapping disks with almost the same surface-surface and
surface-void functions. The values of CFs are almost identical for
high-resolution and magnified images; the only flaw being the difference in
$F_{SS}$ for resized and digitized images due to pixelization of the interface
(the absence of "noise" on $F_{SS}$ for the rescaled image).

\begin{figure*}[!pt]
  \centering
  \subfigure[The image resulting from digitizing of Poisson disks to
    the size of $300 \times 300$ pixels, and then magnified to
    $4000 \times 4000$ pixels using bilinear interpolation]{
    \includegraphics[width=0.25\linewidth, frame]{resize-effects/disks-bilinear.png}
    \label{fig:resized-bilinear}}
  \subfigure[Digitized image of Poisson disks with the size of
    $4000 \times 4000$ pixels]{
    \includegraphics[width=0.25\linewidth, frame]{resize-effects/disks-big.png}
    \label{fig:resized-true}}
  \vskip\baselineskip
  \subfigure[Surface-surface functions]{
    \includegraphics[width=0.45\linewidth]{resize-effects/surfsurf.png}
    \label{fig:resized-ss}}
  \hfill
  \subfigure[Surface-void functions]{
    \includegraphics[width=0.45\linewidth]{resize-effects/surfvoid.png}
    \label{fig:resized-sv}}
  \caption[]{An example of images with almost identical correlation functions
    obtained using different methods: direct digitalization from known disk
    positions and magnification using bilinear interpolation. Image filtering
    was used for interface extraction.}
  \label{fig:resized}
\end{figure*}

\subsection{Criterion for accurate evaluation of surface CFs from discrete images}
\label{sec:crit}
Let us define one-dimensional forward ($F$) and inverse ($F^{-1}$) Fourier
transforms as:
\begin{align}
  \hat{f}(z) &= F[f](z) = \frac{1}{\sqrt{2\pi}}\int_{-\infty}^{\infty} f(x)
  e^{-i\pi xz} dx \label{eq:fourier-forward} \\
  f(x) &= F^{-1}[\hat{f}](z) = \frac{1}{\sqrt{2\pi}}\int_{-\infty}^{\infty} \hat{f}(z)
  e^{i\pi xz} dz \label{eq:fourier-backward}
\end{align}

Here $f(x)$ and $\hat{f}(z)$ can be thought of as representations of a signal in
the time domain and the frequency domain, respectively. Both equations
\cref{eq:fourier-forward} and \cref{eq:fourier-backward} preserve norm on $L_2$:
$(f, f) = (\hat{f}, \hat{f})$. It is said that Fourier transform preserves
``energy'' of the signal. According to Riemann-Lebesgue lemma \cite{HilbertSpaces}
energy of any signal ``concentrates'' around low frequencies. A measure of how
much energy concentrated in low frequencies (for some definition of low
frequencies) is a key to the understanding of the problem of correctness of our
approach. The Shannon sampling theorem \cite{HilbertSpaces} states that it is
possible to reconstruct a band-limited signal $f(x)$ whose frequencies lie in
the range $[0, f]$ from a sequence of samples $\left\{f_n\right\}$ if the
sampling rate is no less than $2f$. We introduce a parameter $C_a$:
\begin{align*}
  f_0(x) &= f(x) - \langle f(x) \rangle \\
  C_a &= \frac{\int_{-a\omega}^{a\omega} |\hat{f_0}(z)|^2
    dz}{\int_{-\omega}^{\omega} |\hat{f_0}(z)|^2 dz}
\end{align*}
where $\omega = 2\pi f$ and $\langle f(x) \rangle$ is the mean value of $f(x)$
over its domain. The criterion for correctness is then:
\begin{equation*}
  C_a > 1 - \xi
\end{equation*}
for some $a$ and $\xi$. This criterion tells us exactly how much energy in
$f_0(x)$ is concentrated in a low frequency range $[0, af]$ compared to the
whole range $[0, f]$. Here we propose $a = 0.5$ and $\xi = 0.07$ as
a strict criterion which is based on results of
\cref{sec:comparison}. For overlapping disks we immediately observe that the
image with resolution $4096 \times 4096$ has surface CFs very close to
analytical solution (\cref{fig:fss-scaling} - \cref{fig:fsv-scaling}) and
perfectly satisfy our $C_{0.5}$ criterion. The other three downscaled images
fail to do so. The first two images of value noise (with
resolution $4096 \times 4096$ and $1024 \times 1024$) pass the criterion and the
other two do not, which is in agreement with \cref{fig:fss-scaling-noise} -
\cref{fig:fsv-scaling-noise} (correlation functions converge at resolution about
$1024 \times 1024$).

Thus, we conclude after inspecting \cref{fig:disks-noise} and
\cref{fig:scaling-alltogether} that the criterion correctly selects images which
are suitable for calculation of surface correlation functions. The criterion
$C_{0.5}$ decays fast with a decline in the image resolution as demonstrated in
\cref{fig:crit-plot}.

\begin{figure}[ht]
  \centering
  \includegraphics[width=\linewidth]{images/plot-criterion1.png}
  \caption[]{Decay of criterion $C_{0.5}$ as a function of the image resolution
    for the downscaled image (shown as a resize ratio). The image used for the
    study is shown as an inset.}
  \label{fig:crit-plot}
\end{figure}

\subsection{The choice of method to extract the interface}
\begin{figure}[ht]
  \centering
  \includegraphics[width=\linewidth]{images/dm_sobel.png}
  \caption{Surface-surface correlation function calculated with different
    interface extraction methods (Edge-detection filter to the left, and
    distance map to the right on the insets). Discretization of each disk is 20
    pixels per diameter, the size is $1000 \times 1000$ pixels.}
  \label{fig:interface-extraction}
\end{figure}

Finally, we explore all major edge-detecting filters, including "naive" distance
map approach. Surface-surface functions for overlapping Poisson disks
($R = 10,\ \lambda = 2 \cdot 10^{-3}$ on image with dimensions of $1000 \times 1000$
pixels) computed with different methods of interface extraction are shown on
\cref{fig:interface-extraction}. These methods include image filtering with
filter \cref{eq:filter-2d}-\cref{eq:filter-3d-3} and interface extraction with
distance map. When using distance map both outer (\cref{eq:dist-outer}) and
inner (\cref{eq:dist-inner}) interfaces are used to compute the correlation
function. An average over the last two functions is also shown. Edge-detecting
filter clearly outperforms the distance map approach. However, the result on
\cref{fig:interface-extraction} could be related to the resolution of studied
image. If we now increase the discretization of the disks and, thus, the image
resolution ($R = 70,\ \lambda = 3 \cdot 10^{-6}$ on image with dimensions of
$5000 \times 5000$ pixels) we get the result as presented on
\cref{fig:sobel-vs-distance-map}. In this case the image filtering method again
collapses to an analytical solution whilst the distance map method failed to do
so. This subsection's results clearly advocate in favor of the edge-detecting
filter in case of surface CFs evolution based on digital images.

To demonstrate the possibilities to compute directional surface correlation
functions, on \cref{fig:direction-vs-map} we show a comparison of the ensemble
$F_{SS}$ computed by averaging the full correlation map against the CF computed
along a single direction (the input image is isotropic Poisson disks from
\cref{fig:interface-extraction}). As expected, less sampling results in more
noise but in general coincide with analytical and ensemble average computed
CFs.

\begin{figure}[ht]
  \centering
  \includegraphics[width=\linewidth]{images/sobel-vs-distance-map.png}
  \caption{Comparison of edge-detecting filter and distance map approach on
    high-resolution image with 140 pixels per disk diameter, the size is
    $5000 \times 5000$ pixels.}
  \label{fig:sobel-vs-distance-map}
\end{figure}

\begin{figure}[ht]
  \centering
  \includegraphics[width=\linewidth]{images/direction_and_map.png}
  \caption{The comparison of directional and ensemble averaged over the full map
    $F_{SS}$.}
  \label{fig:direction-vs-map}
\end{figure}

In addition to edge-detecting filter of our choice, there are plenty of other
kernels for gradient evaluation \cite{bickley1948,prewitt1970,ando2000}. In
total, we investigated Sobel, Ando, Sharr, Bickley and Prewitt filters. Their
performance is almost the same (\cref{fig:kernels}). The errors against the
analytical solution are very similar and the overall performance depends on the
particular image. Therefore, it is difficult to recommend specific kernel; so,
the choice of filter \cref{eq:filter-2d}-\cref{eq:filter-3d-2} is purely due to
the fact that it is computationally more efficient than gradient searching
filters.

\begin{figure}[ht]
  \centering \includegraphics[width=\linewidth]{images/kernels.png}
  \caption{Comparison of different edge-detecting filters in their ability to
    highlight the interface for surface-surface CFs computation (only the case
    of $F_{SS}$ is shown due to similarity of the results for
    $F_{SV}$).}
  \label{fig:kernels}
\end{figure}

\section{Application to binary XCT and SEM images of porous media}
\label{sec:application}
After the verification of our computational approach based on analytical
solution and establishing the criterion $C_{0.5}$ we now have enough tools to
evaluate surface CFs for different porous media images, including real XCT and
SEM data.

\subsection{Porous media images collection}
To test our surface CFs computational framework we collected a variety of 3D XCT
and 2D SEM images of two types of natural porous media: sandstones and carbonate
rocks. The former are expected to have most of their porosity visible on XCT
scans, expect for nano-porosity associated with the clay that can clog the pore
space. Carbonate rocks are monomineral samples, but usually exhibiting
hierarchical structure, i.e., XCT scanning can reveal only pore sizes with the
range within the imaging resolution, whilst imaging with higher resolution
(e.g., SEM or FIB-SEM techniques) usually reveals sub-micron (under-resolution
porosity for XCT) porosity. The relevant details on the samples are provided in
\cref{tab:samples}.

\begin{table*}[!pt]
  \centering
  \begin{ruledtabular}
    \begin{tabular}{|c|c|c|c|}
      Sample name & Image type & Dimensions (pixels) & Image resolution ($\mu m$) \\
      \hline
      Sandstone 1 & XCT & $500 \times 500 \times 500$ & 3.02 \\
      Sandstone 2 & XCT & $500 \times 500 \times 500$ & 3.02 \\
      Sandstone 3 & XCT & $500 \times 500 \times 500$ & 3.02 \\
      Carbonate 1 & XCT & $500 \times 500 \times 500$ & 3.02 \\
      Carbonate 2 & XCT & $500 \times 500 \times 500$ & 3.02 \\
      Carbonate 3 & XCT & $500 \times 500 \times 500$ & 3.02 \\
      Sandstone 4 & SEM &  $1280 \times 869$ & 0.10 \\
      Sandstone 5 & SEM &  $1280 \times 869$ & 0.05 \\
      Carbonate 4 & SEM &  $1024 \times 691$ & 0.06 \\
      Carbonate 5 & SEM &  $1024 \times 691$ & 0.29 \\
      Carbonate B1 & SEM & $1024 \times 691$ & 0.57 \\
      Carbonate B2 & SEM & $1024 \times 691$ & 1.09 \\
      Carbonate B3 & XCT & $500 \times 500 \times 500$ & 1.07 \\
      Carbonate B4 & XCT & $500 \times 500 \times 500$ & 1.07
    \end{tabular}
  \end{ruledtabular}
  \caption{Summary of samples used for calculation of surface correlation
    functions.}
  \label{tab:samples}
\end{table*}

\subsection{Surface correlation functions for the studied collection}
On \cref{fig:real-data-xct} and \cref{fig:real-data-sem} we report computed
$F_{SS}$ and $F_{SV}$ correlation functions for a variety of XCT and SEM images,
respectively. Almost all these cases satisfy our $C_{0.5}$ criterion and, thus,
provide a possibility to interpret resulting CFs.

Images on \cref{fig:real-data-bad} fail to satisfy our $C_{0.5}$ criterion. It is
possible to apply the magnification technique to improve $C_{0.5}$, but this
lacks physical meaning -- from our experience we can immediately tell that these
rocks possess significant under-resolution porosity that would reveal itself on
higher-resolution images. We discuss the strategies for correct evaluation of
surface correlation functions for such porous media in \cref{sec:outline}.

In the general case, the correlation functions of the boundary of a binary set
have at least two characteristic properties: the limit of $F_{ss}$ at large
distances is related to the square of the surface area (in voxels), and the rate
of decrease of the correlation function near zero characterizes the "memory" of
the random process describing the boundary. The last property can be described
by introducing the concept of correlation radius -- the characteristic
attenuation distance of the correlation function. Figure \ref{fig:real-data-xct}
shows examples of XCT images of sandstones and carbonates along with surface
correlation functions calculated using developed methodology.  One can see from
the graphs presented that the right limit for all sandstones is significantly
higher than for carbonate materials, which corresponds to a larger surface area
for these cubes (in voxels). However, the correlation boundary radius for
carbonate samples is larger than for sandstones, which indicates a greater
surface smoothness.

We consider it important to mention one more surface parameter, which can be
easily calculated based on the presented methodology. One of the key topological
properties of rough surfaces for physics and applications is the fractal
dimension. Let us divide the micromodel into cubes with edge length
$\varepsilon$ (in voxel). The topological measure of the surface is defined as
\begin{equation*}
  M_d(\varepsilon) = N(\varepsilon) \varepsilon^d
\end{equation*}
where $N(\varepsilon)$ is the number of cubes covering at least one point of the
surface. The fractal dimension is defined as the value of $d$ for which
$M_d(\varepsilon) \rightarrow const \ne 0$ for $\varepsilon \rightarrow 0$. In
the case of piecewisesmooth line or surface, the fractal dimension is an integer
equal to 1 or 2, respectively (coincides with the topological dimension). For an
arbitrary surfaces, the fractal dimension is not an integer (see for example
work \cite{khlyupin2015fractal}, where fractal dimensions of real rock pore
surfaces are calculated and analysed). The fractal dimension can be effectively
calculated using the box covering algorithm. For each $\varepsilon$, the number
$N(\varepsilon)$ of cubes covering the pore surface is calculated; fractal
dimension $d$ is estimated from the slope of the curve describing
$\ln N(\varepsilon)$ as a function of $\ln \frac{1}{\varepsilon}$. Thus,
calculating the correlation functions $F_{ss}$ of the surface for different
spatial sampling scales $\varepsilon$ (see \cref{fig:scaling-alltogether} with
our results of surface-surface and surface-void correlation functions for images
obtained with different resolution) and analyzing the asymptotic at large
distances gives us the value of $N^2(\varepsilon)$. This makes the further
calculation of the fractal dimension an easy technical procedure.

It has been shown in many works (see for example
\cite{aslyamov2019theoretical,khlyupin2016effects,feder1988}) that the fractal
dimension of a surface as a measure of roughness can make a significant
contribution to such physical phenomena as the thermodynamics of adsorption and
materials storage properties, the formation of thin films, surface diffusion,
and adhesion. However, further development of this issue is not the subject of
this article and requires a separate study, so we only briefly discuss it in
this article as a further possibility.

\begin{figure*}[!pt]
  \centering
  \subfigure[Sandstone 1, $C_{0.5} = 0.938$]{
    \includegraphics[width=0.22\linewidth]{images/sandstone1.png}
    \label{fig:sandstone1}}
  \hfill
  \subfigure[Sandstone 2, $C_{0.5} = 0.937$]{
    \includegraphics[width=0.22\linewidth]{images/sandstone2.png}
    \label{fig:sandstone2}}
  \hfill
  \subfigure[Sandstone 3, $C_{0.5} = 0.939$]{
    \includegraphics[width=0.22\linewidth]{images/sandstone3.png}
    \label{fig:sandstone3}}
  \vskip\baselineskip
  \subfigure[Carbonate 1, $C_{0.5} = 0.950$]{
    \includegraphics[width=0.22\linewidth]{images/carbonate1.png}
    \label{fig:carbonate1}}
  \hfill
  \subfigure[Carbonate 2, $C_{0.5} = 0.954$]{
    \includegraphics[width=0.22\linewidth]{images/carbonate2.png}
    \label{fig:carbonate2}}
  \hfill
  \subfigure[Carbonate 3, $C_{0.5} = 0.946$]{
    \includegraphics[width=0.22\linewidth]{images/carbonate3.png}
    \label{fig:carbonate3}}
  \vskip\baselineskip
  \subfigure[Surface-surface function]{
    \includegraphics[width=0.475\linewidth]{images/XCT-ss.png}
    \label{fig:plot-ss-real}}
  \hfill
  \subfigure[Surface-void function]{
    \includegraphics[width=0.475\linewidth]{images/XCT-sv.png}
    \label{fig:plot-sv-real}}
  \caption[]{Examples of XCT images along with surface correlation functions
    calculated using developed methodology.}
  \label{fig:real-data-xct}
\end{figure*}

\begin{figure*}[!pt]
  \centering
  \subfigure[Sandstone 4, $C_{0.5} = 0.931$]{
    \includegraphics[width=0.22\linewidth, frame]{images/846-79_0009.png}
    \label{fig:sandstone4}}
  \hfill
  \subfigure[Sandstone 5, $C_{0.5} = 0.951$]{
    \includegraphics[width=0.22\linewidth, frame]{images/859-12_0005.png}
    \label{fig:sandstone5}}
  \hfill
  \subfigure[Carbonate 4, $C_{0.5} = 0.931$]{
    \includegraphics[width=0.22\linewidth, frame]{images/C1-1-g6_010.png}
    \label{fig:carbonate4}}
  \hfill
  \subfigure[Carbonate 5, $C_{0.5} = 0.931$]{
    \includegraphics[width=0.22\linewidth, frame]{images/C3-3-G0 (32).png}
    \label{fig:carbonate5}}
  \vskip\baselineskip
  \subfigure[Surface-surface function]{
    \includegraphics[width=0.475\linewidth]{images/SEM-ss.png}
    \label{fig:plot-ss-real-sem}}
  \hfill
  \subfigure[Surface-void function]{
    \includegraphics[width=0.475\linewidth]{images/SEM-sv.png}
    \label{fig:plot-sv-real-sem}}
  \caption[]{Examples of surface correlation functions evaluated for SEM images
    of real porous media.}
  \label{fig:real-data-sem}
\end{figure*}

\begin{figure*}[!pt]
  \centering
  \subfigure[Carbonate B1, $C_{0.5} = 0.841$]{
    \includegraphics[width=0.25\linewidth, frame]{images/C6-1-2-G2_019.png}
    \label{fig:carbonate-b1}}
  \subfigure[Carbonate B2, $C_{0.5} = 0.815$]{
    \includegraphics[width=0.25\linewidth, frame]{images/C6-1-2-G2_024.png}
    \label{fig:carbonate-b2}}
  \vskip\baselineskip
  \subfigure[Carbonate B3, $C_{0.5} = 0.912$]{
    \includegraphics[width=0.25\linewidth]{images/carbRNF1_500.png}
    \label{fig:carbonate-b3}}
  \subfigure[Carbonate B4, $C_{0.5} = 0.888$]{
    \includegraphics[width=0.25\linewidth]{images/carbRNF2_500.png}
    \label{fig:carbonate-b4}}
  \caption[]{Examples of SEM and XCT images which do not satisfy our $C_{0.5}$
    criterion.}
  \label{fig:real-data-bad}
\end{figure*}

\subsection{Computational efficiency}
\label{sec:compute}
We measured the execution times of functions from our
\code{CorrelationFunctions.jl} package to compute $F_{SS}$ and $F_{SV}$. The
hardware we used is Intel Xeon Gold 6248R CPU with DDR4 2400 MHz RAM and Nvidia
GeForce RTX 3090 GPU running under Linux. Input data consists of 2D and 3D
square bit arrays with side varying from 1000 to 10000 with step 1000 (2D case)
or from 100 to 1000 with step 100 (3D case). The obtained wall times are
presented on \cref{fig:exec-times}. As our video RAM is limited, the GPU can
process three dimensional arrays with dimensions no more than $700 \times 700
\times 700$ voxels. A observed speedup from the use of GPU is about 30 times as
compared to CPU computations.

\begin{figure}[ht]
  \centering
  \includegraphics[width=\linewidth]{images/exec-time.png}
  \caption[]{Execution times for 2D and 3D input images for CPU and GPU
    implementations.}
  \label{fig:exec-times}
\end{figure}

\section{Discussion and outline}
\label{sec:outline}
The "digital" approach for applications to real XCT and SEM images as developed
in this paper has some important limitations. Namely, the magnification-based
correction of coarse images (\cref{sec:magnify}) or the $C_{0.5}$ criterion
(\cref{sec:crit}) provide accurate surface correlation functions only in case
the under-resolution porosity is absent in the image. This means that accurate
CFs can be sampled only for the case that with increasing resolution no new
details appear on the image. The ideal imaging resolution is not known a priory
and, thus, can not be assessed with $C_{0.5}$. For sandstones, we expect
accurate surface CFs evaluation, but for carbonates on \cref{fig:real-data-bad}
the presence of unresolved porosity is expected. For other carbonates such as
Carbonates 1-3 the presence of such porosity is likely, and for Carbonates 4-5
the resolution is probably optimal -- however, both of these hypotheses needs to
be tested by imaging with a higher resolution. Another way would be to compare
"digital" CFs with those obtained experimentally using small angle scattering
(note that current generation synchrotrons are operating in the limit
$\lesssim 1 \mu$ resolution). On the other hand, the surface CFs we obtain are
correct under the assumption of no sub-resolution details. This is an important
achievement, as the problem of under-resolution porosity can be then solved by
multi-scale image analysis \cite{moctezuma2002,SciRep1}.

Considering the accuracy of the magnification procedure, it is also closely
connected to the under-resolution porosity problem. On \cref{fig:resized}
it was clear that magnification can result in correct values of surface CFs, but
for $F_{SS}$ some important (sub-resolution) information is simply not
available. We argue that this unfortunate downside can be potentially mitigated
by higher-resolution imaging of the interface in question with subsequent
correction of the magnified surface-surface function. Another interesting
approach would be to relate the $F_{SS}$ to some roughness coefficient that
could change with time due to, for example, reactive transport within the pores
\cite{godinho2016,noiriel2021,prokhorov2022} or due to mechanical load. Thus, in
conjunction with high-resolution imaging $F_{SS}$ and $F_{SV}$ could serve as
excellent descriptors of surface change dynamics \cite{chen2022}.

We described two ways to extract the interface to compute surface CFs -- "naive"
approach based on distance map transform, and using edge-detecting filter. While
we showed that the latter approach is much more accurate (e.g.,
\cref{fig:sobel-vs-distance-map}), the naive one is still useful. In case we
want to utilize $F_{SS}$ and $F_{sSV}$ for stochastic reconstructions
\cite{Y-T,JiaoPNAS,karsaninaPRL} we advocate the usage of distance map
transform; moreover, the choice of inner or outer interface
(\cref{fig:interface-extraction}) is not important. The reason is the procedure
of CFs update during the simulated annealing purpose. The optimization of the
filter-based approach is more involved and requires updates of the filter within
some area first, and then local CFs recalculations. On the other hand, the naive
approach allows to utilize the classical straightforward and extremely efficient
technique \cite{rozman2001}. Considering that the exact values of the CFs are
not important, the usage of distance maps introduces no inconsistencies, but
improves the results of the reconstruction.

In addition to fundamental XCT and SEM images acquisition problems (such as
inhomogeneous chemical composition and density of the solid phase, and partial
volume effects), as discussed in \cref{sec:intro}, the "digital" approach to
evaluate surface correlation functions enjoys some additional advantages. As
\cref{sec:compute} demonstrated, digital computations can be fast -- this can be
achieved by FFT on modern GPU hardware, or by vectorized linear segment scanning
\cite{EPL1}. In addition to efficient computations, digital methods allow
efficient computations of higher order statistics, such as n-point correlation
functions \cite{malmir2018}. Such efficiency and flexibility in computations is
not available for "continuous" approach. This statement is by no means a
critique of such an approach, as it allows to compute the exact surface CFs in
cases when the structure can be described by some continuous function, e.g.,
Gaussian random field \cite{ma2018SS}. Aforementioned computational efficiency
opens numerous way to apply surface correlation functions in description of real
surfaces, surface structural dynamics and establishes them as universal
interface descriptors. As such, $F_{SS}$ and $F_{SV}$ can serve as metrics in
deep learning application for stochastic reconstructions and especially
super-resolution, e.g., \cite{chen2020super,janssens2020,karimpouli2022}, or can
be, likely, even incorporated into the cost function during training on modern
GPUs.

\section{Summary}
\label{sec:summary}
In this contribution we proposed a computationally efficient and very flexible
"digital" framework to evaluate surface correlation functions for XCT and SEM
images of real porous media and materials. The methodology not only allows to
compute the full correlation maps in addition to directional or ensemble average
CFs, but also does this on both CPU and GPU architectures. The code to perform
all computations is fully available as a Jupiter notebook (see Supplementary
Materials); moreover, the full functionality is now a part of the freely
available \code{CorrelationFunctions.jl} package. Based on the computed $F_{SS}$
and $F_{SV}$ functions for a variety of artificial and real XCT/SEM digital
images we were able to conclude:
\begin{enumerate}
  \item On digital images edge-detecting filters can be utilized to effectively
    extract the interface for surface CFs computations.
  \item It is possible to use digital images as a foundation for robust surface
    correlation functions evaluation in case their spatial resolution is
    adequate.
  \item It is possible to establish $C_{0.5}$ criterion to evaluate if the
    quality of the image at hand is adequate; for artificial images passing this
    criterion the analytically known and computed CFs are similar.
  \item In case the $C_{0.5}$ criterion is below the desired value, one can
    utilize image magnification (e.g., in the form of bicubic interpolation) to
    increase the image resolution and achieve the necessary $C_{0.5}$ criterion
    -- in this latter case the surface CFs computed from the magnified image
    reach high accuracy.
\end{enumerate}
The developed methodology allows applying surface CFs to describe the structure
of porous materials based on their experimental images and enhance stochastic
reconstructions or super-resolution procedures, or serve as an efficient metrics
in machine learning applications.

\section{Acknowledgments}
This research was supported by the Russian Science Foundation grant
19-72-10082.

Collaborative effort of the authors within the FaT iMP (Flow and Transport in
Media with Pores) research group (www.porenetwork.com) and used some of its
software. We thank Prof. Salvatore Torquato for directing us to a
derivation of analytical formulas for 2D Poisson disks. Special thanks go to
our colleague Dr. Dina Gafurova for the 2D and 3D images of real porous media
used in this study.

\appendix
\section{Derivation of analytical solutions for 2D Poisson disks}
\label{ap:overlapping-disks}
Analytic representation of surface-surface and surface-void correlation
functions for overlapping balls with centers generated by Poisson point process
is well known \cite{Torquato_book}:
\begin{align}
  F_{SV}(r) &= -\lim_{a_1 \rightarrow R} \frac{\partial}{\partial a_1}
  e^{-\lambda S_{tot}(r, a_1, R)} \label{eq:fsv-disks} \\
  F_{SS}(r) &= \lim_{a_1, a_2 \rightarrow R} \frac{\partial}{\partial a_1}
  \frac{\partial}{\partial a_2} e^{-\lambda S_{tot}(r, a_1,
    a_2)} \label{eq:fss-disks}
\end{align}
Here $S_{tot}(r, a_1, a_2)$ refers to a common volume of two n-dimensional balls
of radii $a_1$ and $a_2$ with a distance $r$ between their centers and $\lambda$
is a parameter of Poisson process. For two-dimensional disks we have the
following expression for $S_{tot}(r, a1, a2)$:
\begin{equation}
  S_{tot}(r, a_1, a_2) = \pi a_1^2 + \pi a_2^2 - S_{int}(r, a_1, a_2) \label{eq:total}
\end{equation}
where $S_{int}(r, a_1, a_2)$ is a common area of two disks, being equal to:
\begin{align}
  S_{int}(r, a_1, a_2) =&  a_1^2 \arccos(\frac{r^2+a_1^2-a_2^2}{2a_1r}) + \\
  & a_2^2 \arccos(\frac{r^2+a_2^2-a_1^2}{2a_2r}) - \\
  & \frac{\sqrt{Y}}{2} \label{eq:intersection}
\end{align}
when $r<2R$ and zero otherwise. $Y$ in \cref{eq:intersection} is
\begin{equation*}
  Y = (-r+a_1+a_2)(r+a_2-a_1)(r+a_1-a_2)(r+a_1+a_2)
\end{equation*}
Substituting \cref{eq:intersection} into \cref{eq:total} and then
\cref{eq:total} into \cref{eq:fss-disks} we obtain \cref{eq:fss_final} for
$F_{SS}(r)$.

Similarly substituting \cref{eq:intersection} into \cref{eq:total} and then
\cref{eq:total} into \cref{eq:fsv-disks} we obtain \cref{eq:fsv_final} for
$F_{SV}(r)$.

\section{Implementation of surface function computations}
\label{ap:b_code}
The code used to compute surface correlation functions in this manuscript was
written in Julia language and available as a Jupyter notebook in Supplementary
Materials. It uses \code{CorrelationFunctions.jl} package developed by our group
\cite{CFsjlpaper} that allows efficient computation of surface and other
correlation functions using both CPU and GPU architectures. The package allows
to compute all classical CFs described in Torquato’s book \cite{Torquato_book}
from digital images.

\bibliography{apssamp}
\end{document}
